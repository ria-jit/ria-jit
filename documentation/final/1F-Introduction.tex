\subsection{Problembeschreibung} % Noah
\begin{frame}
    \frametitle{Problembeschreibung}

    \vspace{0.50cm}

    \textbf{RISC--V:} Offene ISA, die dem Reduced Instruction Set Computer (RISC) Schema folgt.

    \vspace{0.50cm}

%todo single threaded, user space @noah
    \textbf{Problem:}
    \begin{itemize}
        \item RISC--V Prozessoren sind noch nicht weit verbreitet.
        \item Entwickler, die Code für RISC--V als Zielplatform kompilieren wollen, können diesen nicht nativ ausführen.
    \end{itemize}

    \vspace{0.50cm}

    \textbf{Lösung:} Emulieren des RISC--V Befehlsatzes auf einem x86--64 Prozessor

    \vspace{0.50cm}

    \begin{block}{Warum x86--64?}
        x86--64 ist der derzeitige Standard für Prozessoren in Laptops und Desktop-PCs.
    \end{block}
\end{frame}

\subsection{RISC--V vs. x86--64} % Noah
\begin{frame}
    \frametitle{RISC--V vs. x86--64}
    \framesubtitle{Gegenüberstellung}

    \begin{minipage}[t]{.47\textwidth}
        \textbf{RISC--V Übersicht:}
        \vspace{0.20cm}
        \begin{itemize}
            \item RISC Schema
            \item Load-Store-Architektur
            \item 31 General Purpose Register
            \item 32 Floating Point Register
            \item 3-Operanden Adressform
            \item Spezielles Zero-Register
        \end{itemize}
    \end{minipage}
    \begin{minipage}[t]{.47\textwidth}
        \textbf{x86--64 Übersicht:}
        \vspace{0.20cm}
        \begin{itemize}
            \item CISC Schema
            \item Register-Memory-Architektur
            \item 16 General Purpose Register
            \item 16 Floating Point (\texttt{XMM}) Register
            \item 2-Operanden Adressform
        \end{itemize}
    \end{minipage}
\end{frame}

\subsection{Dynamische Binärübersetzung} % Noah
\begin{frame}
    \frametitle{Instruction Set Emulation}
    \framesubtitle{Interpretation vs. dynamische Übersetzung}
    \begin{columns}[c,onlytextwidth]
        \begin{column}{0.45\textwidth}
            \begin{exampleblock}{Vorteile}
                \begin{itemize}
                    \item[$\textcolor{TUMGreen}\blacksquare$] Einfach zu implementieren
                    \item[$\textcolor{TUMGreen}\blacksquare$] Keine Erzeugung von JIT Assembler nötig
                \end{itemize}
            \end{exampleblock}

            \vspace{0.50cm}

            \begin{alertblock}{Nachteile}
                \begin{itemize}
                    \item[$\textcolor{TUMOrange}\blacksquare$] Mehrfache Übersetzung der selben Instruktion
                    \item[$\textcolor{TUMOrange}\blacksquare$] Wenig Optimierungspotential
                \end{itemize}
            \end{alertblock}
        \end{column}
        \begin{column}{0.45\textwidth}
            \begin{exampleblock}{Vorteile}
                \begin{itemize}
                    \item[$\textcolor{TUMGreen}\blacksquare$] Einfach zu implementieren
                    \item[$\textcolor{TUMGreen}\blacksquare$] Keine Erzeugung von JIT Assembler nötig
                \end{itemize}
            \end{exampleblock}

            \vspace{0.50cm}
        \end{column}
    \end{columns}
\end{frame}

% todo @Noah rework frame titles
\begin{frame}
    \frametitle{Dynamische Binärübersetzung}
    \framesubtitle{Statische Binärübersetzung}

    \textbf{Vorgehen}
    \begin{itemize}
        \item Eimaliges Übersetzen der gesamten Auszuführenden Datei
        \item Ausführen der Übersetzung
    \end{itemize}

    \vspace{0.50cm}

    \textbf{Vorteile}
    \begin{itemize}
        \item Einmaliger Übersetzungsaufwand
        \item Theoretisch fast native Geschwindigkeit erreichbar
    \end{itemize}

    \vspace{0.50cm}

    \textbf{Nachteile}
    \begin{itemize}
        \item Schwierig zu implementieren (Halteproblem)
        \item Problematisch bei Änderungen der Assembly zur Runtime
    \end{itemize}
\end{frame}

\begin{frame}
    \frametitle{Dynamische Binärübersetzung}
    \framesubtitle{Dynamische Binärübersetzung}

    \textbf{Vorgehen}
    \begin{itemize}
        \item Einlesen eines Blocks von Befehlen
        \item (Optional) Optimierungen
        \item Übersetzen des einzelnen Blocks
        \item Ausführen des Übersetzten Blocks
    \end{itemize}

    \vspace{0.50cm}

    \textbf{Vorteile}
    \begin{itemize}
        \item Nur tatsächlich notwendige Teile des Programms werden übersetzt
        \item Instruktionen werden nur einmal Übersetzt
        \item Blöcke von Instruktionen eigenen sich zur Optimierung
    \end{itemize}

    \vspace{0.50cm}

    \textbf{Nachteile}
    \begin{itemize}
        \item Problematisch bei Änderungen der Assembly zur Runtime
    \end{itemize}
    %todo spacing? fußzeile etwas nah!
\end{frame}