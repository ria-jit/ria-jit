\subsection{Problembeschreibung} % Noah
\begin{frame}
    \frametitle{Problembeschreibung}

    \vspace{0.50cm}

    \textbf{RISC--V:} Offene ISA die dem Reduced Instruction Set Computer (RISC) Schema folgt.

    \vspace{0.50cm}

    \textbf{Problem:}
    \begin{itemize}
        \item Verfügbarkeit von RISC--V Prozessoren ist begrenzt.
        \item Entwickler die Code für RISC--V als Zielplatform compilieren können diesen nicht ausführen.
    \end{itemize}

    \vspace{0.50cm}

    \textbf{Lösung:} Emulieren des RISC--V Befehlsatz auf einem x86--64 Prozessor

    \vspace{0.50cm}

    \begin{block}{Warum x86--64?}
        x86--64 ist der derzeitige Standard für Prozessoren in Laptops und Desktop-PCs.
    \end{block}
\end{frame}

\subsection{RISC--V vs. x86--64} % Noah
\begin{frame}
    \frametitle{RISC--V vs. x86--64}
    \framesubtitle{Gegenüberstellung}

    \begin{minipage}[t]{.47\textwidth}
        \textbf{RISC--V Übersicht:}

        \begin{itemize}
            \item RISC Schema
            \item Load-Store-Architektur
            \item 31 General Purpose Register
            \item 32 Floating Point Register
            \item 3-Operanden Adressform
            \item Spezielles Zero-Register
        \end{itemize}
    \end{minipage}
    \begin{minipage}[t]{.47\textwidth}
        \textbf{x86--64 Übersicht:}

        \begin{itemize}
            \item CISC Schema
            \item Register-Memory-Architektur
            \item 16 General Purpose Register
            \item 16 Floating Point (\texttt{XMM}) Register
            \item 2-Operanden Adressform
        \end{itemize}
    \end{minipage}
\end{frame}

\subsection{Dynamische Binärübersetzung} % Noah

