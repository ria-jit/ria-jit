\documentclass[german]{tum-presentation}

\usepackage[utf8]{inputenc}
\usepackage{csquotes}
\usepackage[T1]{fontenc}
\usepackage{amsmath}
\usepackage{amsfonts}
\usepackage{amssymb}
\usepackage{float}
\usepackage{graphicx}
\usepackage{booktabs}
\usepackage{icomma}
\usepackage{siunitx}
\usepackage{multicol}
\usepackage[german]{varioref}

\newcommand*\diff{\mathop{}\!\mathrm{d}}
\newcommand*\Diff[1]{\mathop{}\!\mathrm{d^#1}}
\newcommand{\define}[2]{\item \textbf{#1}\\#2}
\newcommand{\conclude}[0]{\ensuremath{\Longrightarrow} }
\newcommand{\refer}[0]{\ensuremath{\rightarrow} }
\sisetup{range-phrase=--,range-units=single}
\MakeOuterQuote{"}

\addbibresource{literature.bib}

\title[Binary Translation: RISC--V \refer x86--64]{Dynamische Binärübersetzung: RISC--V \refer x86--64}
\subtitle{Zwischenpräsentation}
\author[Dormann, Kammermeier, Pfannschmidt, Schmidt]{Noah Dormann\inst{1}, Simon Kammermeier\inst{1},\\Johannes Pfannschmidt\inst{1}, Florian Schmidt\inst{1}}
\institute[]{\inst{1} Fakultät für Informatik,
  Technische Universität München (TUM)}
\date{21. Juli 2020}

\footline{\insertshortauthor~|~\insertshorttitle}

\setbeamertemplate{section in toc}[sections numbered]
\setbeamertemplate{subsection in toc}[subsections numbered]

\begin{document}

\begin{frame}[noframenumbering]
 	\titlepage
\end{frame}

\begin{frame}
	\frametitle{Gliederung}
	\tableofcontents
\end{frame}

% selbstverständlich sample-Gliederung
\section{Einführung} % Noah
\subsection{Dynamische Binärübersetzung}
\subsection{Grobüberblick über die RISC--V ISA}
\subsection{Angebot}

\section{Systemarchitektur}
\subsection{ELF-Loader} % Simon
\subsection{Parser} % Noah
\subsection{Block loader} % Johannes
\subsection{Code generator} % Flo
\subsection{Code cache} % Flo

\begin{frame}
	\frametitle{Folie}
	Hier könnte Ihre Werbung stehen\footfullcite{riscv-spec}
\end{frame}

\section{Anhang}
\begin{frame}
 	\frametitle{Literaturverzeichnis}
 	\printbibliography
\end{frame}
\end{document}
