\documentclass{article}

\usepackage[ngerman]{babel}
\usepackage[utf8]{inputenc}
\usepackage{fancyhdr}
\usepackage{extramarks}
\usepackage{amsmath}
\usepackage{amsthm}
\usepackage{amsfonts}
\usepackage{ifsym}
\usepackage{tikz}
\usepackage{marvosym}
\usepackage[plain]{algorithm}
\usepackage{algpseudocode}

\newcommand{\titleOfDocument}[0]{Dynamic Binary Translator: RISC--V \refer x86}
\renewcommand{\paragraph}[1]{\vspace{7pt}\noindent\textbf{#1}\newline\nopagebreak\vspace{1pt}}
\newcommand*\diff{\mathop{}\!\mathrm{d}}
\newcommand*\Diff[1]{\mathop{}\!\mathrm{d^#1}}
\newcommand{\define}[2]{\item \textbf{#1}\\#2}
\newcommand{\conclude}[0]{$\Longrightarrow$ }
\newcommand{\refer}[0]{$\rightarrow$ }
\renewcommand\headrulewidth{0.4pt}
\renewcommand\footrulewidth{0.4pt}

\topmargin=-0.45in
\evensidemargin=0in
\oddsidemargin=0in
\textwidth=6.5in
\textheight=9.0in
\headsep=0.25in
\linespread{1.1}

\pagestyle{fancy}
\lhead{ERA-GP: \titleOfDocument}
\rhead{Angebot}
\cfoot{}
\rfoot{\thepage}


\title{
    \vspace{2in}
    \textmd{\textbf{Angebot\\\titleOfDocument}}\\
    \normalsize\vspace{0.1in}\small{\today}\\
    \vspace{0.1in}\large{\textit{Bachelor-Praktikum Rechnerarchitektur (IN0005)\\Großpraktikum}}
    \vspace{2in}
    \centering\\
    \includegraphics[width=4.5cm]{images/tum}
}

\author{Noah Dormann, Simon Kammermeier,\\Johannes Pfannschmidt, Florian Schmidt}
\date{}



\begin{document}
\maketitle
\thispagestyle{empty}
\pagebreak

\section*{Minimalanteil}
\begin{itemize}
	\item Einfache RISC-V-kompilierte Programme ausführen
	\item Unit-Testing der Systemkomponenten
	\item Einfache Syscalls unterstützen (\verb!write!, \verb!read!, \verb!printf!, \verb!open!, \verb!close!, \verb!fstat!, \ldots)
	\item Benchmark GZIP ausführen
	\item Dokumentation anfertigen
\end{itemize}

\section*{Erweiterungsteil}
\begin{itemize}
	\item Benchmarks (Image-Processing) ausführen
	\item Floating Point Extension
	\item Optimierungen (\refer Chaining, return address prediction)
\end{itemize}

\section*{Zeitplan}
\begin{itemize}
	\item Ausführen eines \emph{Hello-World-Programmes} bis zum 10. Juni 2020
	\item Abschluss des \emph{Minimalanteils} (s.o.) bis Mitte Juli
	\item Anfertigung der \emph{Dokumentation} bis Mitte August
\end{itemize}
\end{document}
